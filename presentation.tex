\documentclass{beamer}
\usepackage[utf8]{inputenc}
\usepackage[english,russian]{babel}
\usetheme{Madrid}
\usepackage{graphicx}
\graphicspath{{images/}}

\title{Разработка когнитивных способов представления спектральных данных атомных систем на основе интерактивной компьютерной графики}
\author{Тихонова А. А. \\
        Научный руководитель: Казаков В. В.}
\institute{Новосибирский Государственный Университет}

\makeatletter
\setbeamertemplate{footline}
{
  \leavevmode%
  \hbox{%
  \begin{beamercolorbox}[wd=.5\paperwidth,ht=2.25ex,dp=1ex,left]{title in head/foot}%
    \usebeamerfont{title in head/foot} a.tikhonova@g.nsu.ru
  \end{beamercolorbox}%
  \begin{beamercolorbox}[wd=.5\paperwidth,ht=2.25ex,dp=1ex,right]{date in head/foot}%
    \usebeamerfont{date in head/foot}\insertshortdate{}\hspace*{2em}
    \insertframenumber{} / \inserttotalframenumber\hspace*{2ex} 
  \end{beamercolorbox}}%
  \vskip0pt%
}
\makeatother

\begin{document}

\begin{frame}
\titlepage
\end{frame}

\begin{frame}
\frametitle{Cпектральные данные атомных систем}
    Использование спектральных данных:
    \begin{itemize}
        \item исследование продуктов сгорания
        \item исследования астрофизики, геологии, экологии
        \item физика лазеров
    \end{itemize}
    Способы представления:
    \begin{itemize}
        \item табличный
        \item графический (схема энергетических уровней и переходов)
    \end{itemize}
\end{frame}

\begin{frame}
\frametitle{Эволюция графического представления спектральных данных}
    \begin{itemize}
        \item Спектрограммы
        \item Графическое изображение стационарных состояний по Н. Бору (1923г.)
        \item Схема термов и переходов по Д. С. Рождественскому (1921г.)
        \item Диаграммы Гротриана, В. Гротриан (1924г.)
        \item ИС по атомной спектроскопии, позволяющие на основе БД строить графические представления атомных систем (ASD NIST и ИС "ЭСА"(ИАиЭ))
    \end{itemize}
\end{frame}

\begin{frame}{Эволюция графического представления спектральных данных}{Диаграммы Гротриана}
\begin{columns}
    \begin{column}{0.5\textwidth}
    \begin{block}{}
         Горизонтальная ось - классификация сост. по l и j (энергетические лестницы)
         \\~\\Мультиплетные спектры - каждая ступень распадается на несколько близких ступеней
    \end{block}
    \end{column}
    \begin{column}{0.4\textwidth}
      \includegraphics[width=\textwidth]{grot.png}
    \end{column}
  \end{columns}
     
  
\end{frame}

\begin{frame}
\frametitle{Цели и задачи}
    Цель: разработка новых способов графического представления электронной структуры атомных систем.
     \\~\\
    Задачи:
    \begin{itemize}
        \item Анализ существующих графических представлений и выявление их недостатков
        \item Разработка оригинального графического инструмента для анализа атомных спектров
        \item Добавения возможностей взаимодействия пользователя с диаграммой
        \item Анализ особенностей представления спектральных данных на новой диаграмме
    \end{itemize}
\end{frame}

\begin{frame}
\frametitle{Недостатки классических графических представлений атомных спектров}
    Спектрограммы: представлены только переходы без энергетических уровней, линии тонкого расщепления сливаются
     \\~\\
    Диаграммы Гротриана: потеря читаемости при больших Z – большое число уровней и пересечений спектральных линий
\end{frame}

\begin{frame}
\frametitle{Квантограмма - оригинальная диаграмма спектральных данных}
    Основная идея: переходы отображаются точками на плоскости, координаты которых - значения энергии верхнего и нижнего уровней перехода.
    \\~\\
     \begin{itemize}
        \item Автоматическая генерация на основе БД для всех элементов периодической таблицы и их ионов
        \item Отображение мультиплетности, интенсивности и длины волны перехода
        \item Информация об уровнях
    \end{itemize}
\end{frame}

\begin{frame}
\frametitle{Реализация}
    \begin{itemize}
        \item Трехуровневая клиент-серверная архитектура (MS SQL + Apache и php скрипты + WEB браузер)
        \item Интерактивное web-приложение (Rich Internet Application)
        \item Web-интерфейс: JavaScript, HTML, CSS, SVG
        \item Библиотеки: Chart.js, JQuery
    \end{itemize}
    \begin{columns}
    \begin{column}{0.7\textwidth}
       Вид квантограммы:
    \begin{itemize}
        \item 2D - диаграмма
        \item Переход - точка на графике, отображаемая символом
        \item Символ перехода соответствует мультиплетности, 
        \\~цвет - длине волны
    \end{itemize}
    \end{column}
    \begin{column}{0.25\textwidth}
      \includegraphics[width=\textwidth]{mult.png}
    \end{column}
  \end{columns}
\end{frame}

\begin{frame}{Интерактивность квантограмм}{Масштабирование и всплывающие подсказки}
    \begin{columns}
    \begin{column}{0.5\textwidth}
         Масштабирование
         \includegraphics[width=\textwidth]{zoom.png}
    \end{column}
    \begin{column}{0.5\textwidth}
      Tooltip
      \includegraphics[width=\textwidth]{tooltip.png}
    \end{column}
  \end{columns}
\end{frame}

\begin{frame}{Интерактивность квантограмм}{Режимы отображения данных}
\begin{columns}
    \begin{column}{0.5\textwidth}
    \begin{itemize}
        \item X: E$_{lower}$, Y: E$_{upper}$
        \item  X: E$_{lower}$, Y: E$_{upper}$ - E$_{lower}$
    \end{itemize}
         \includegraphics[width=\textwidth]{mode1.png}
    \end{column}
    \begin{column}{0.5\textwidth}
      \begin{itemize}
        \item Для четного терма - \\*X: E$_{lower}$, Y: E$_{upper}$
        \\*для нечетного - наоборот
    \end{itemize}
      \includegraphics[width=\textwidth]{mode2.png}
    \end{column}
  \end{columns}
\end{frame}

\begin{frame}{Интерактивность квантограмм}{Другие функции}
   \begin{columns}
    \begin{column}{0.25\textwidth}
        \begin{block}{}
             \scriptsize\begin{itemize}
                \item Интенсивность перехода - прозрачность\\~\\
                \item Фильтрация переходов по длине волны\\~\\
                \item см$^{-1}$, эВ\\~\\
                \item Отображение компонент мультиплета в отдельном окне
            \end{itemize}
            \normalsize
        \end{block}
    \end{column}
    \begin{column}{0.75\textwidth}
    \includegraphics[width=\textwidth]{other.png}
    \end{column}
  \end{columns}
\end{frame}

\begin{frame}{Интерактивность квантограмм}{Отображение мультиплетов}
\begin{columns}
    \begin{column}{0.4\textwidth}
        Раскрашивание переходов по принадлежности к мультиплетам
    \end{column}
    \begin{column}{0.6\textwidth}
    \includegraphics[width=\textwidth]{multview.png}
    \end{column}
  \end{columns}
\end{frame}

\begin{frame}{Интерактивность квантограмм}{Ширина конфигурации}
\begin{columns}
    \begin{column}{0.4\textwidth}
        Ширина конфигурации - разница энергий самого высокого и низкого уровней, имеющих данную конфигурацию
        \\~\\
        Квантограмма атома Hg I
    \end{column}
    \begin{column}{0.6\textwidth}
    \includegraphics[width=\textwidth]{config.png}
    \end{column}
  \end{columns}
\end{frame}

\begin{frame}
\frametitle{Диаграмма уровней}
Диаграмма уровней атома Mn
    \includegraphics[width=\textwidth]{lvl.png}
\end{frame}

\begin{frame}
\frametitle{Диаграмма уровней}
    \includegraphics[width=\textwidth]{lvl2.png}\\\begin{center} {$\Downarrow$} \end{center}
    \includegraphics[width=\textwidth]{lvl3.png}
\end{frame}

\begin{frame}{Результаты}{Разделение спектра на серии на примере атома H}
    \includegraphics[width=\textwidth]{H.png}
\end{frame}
\begin{frame}{Результаты}{Щелочные металлы (Na)}
\begin{columns}
    \begin{column}{0.35\textwidth}
    \begin{itemize}
                \item Разделение на серии - изменение L
                \item Дублетная структура
        \end{itemize}
    \end{column}
    \begin{column}{0.65\textwidth}
    \includegraphics[width=\textwidth]{Na.png}
    \end{column}
  \end{columns}
\end{frame}

\begin{frame}{Результаты}{Тонкая структура}
Спектр атома Mn
\begin{columns}
    \begin{column}{0.5\textwidth}
    \includegraphics[width=\textwidth]{Mn2.png}
    \end{column}
    \begin{column}{0.5\textwidth}
    Элементы мультиплета атома Mn
    \includegraphics[width=\textwidth]{Mn3.png}
    \end{column}
  \end{columns}
\end{frame}

\begin{frame}
\frametitle{Итоги}
    \begin{itemize}
        \item Предложен и реализован принципиально новый тип визуализации электронной структуры атома – «квантограмма»

        \item Новый вид диаграммы интегрирован в ИС "ЭСА"
    \end{itemize}
\end{frame}

\end{document}
