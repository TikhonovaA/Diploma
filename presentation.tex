\documentclass{beamer}
\usepackage[utf8]{inputenc}
\usepackage[english,russian]{babel}
\usetheme{Madrid}
\usepackage{graphicx}
\graphicspath{{imgs/}}

\title{Разработка когнитивных способов представления спектральных данных атомных систем на основе интерактивной компьютерной графики}
\author{Тихонова А. А. \\
        Научный руководитель: Казаков В. В.}
\institute{Новосибирский Государственный Университет}

\makeatletter
\setbeamertemplate{footline}
{
  \leavevmode%
  \hbox{%
  %\begin{beamercolorbox}[wd=.333333\paperwidth,ht=2.25ex,dp=1ex,center]{author in head/foot}%
  %  \usebeamerfont{author in head/foot}Каня Кирилл, НГУ
  %\end{beamercolorbox}%
  \begin{beamercolorbox}[wd=.5\paperwidth,ht=2.25ex,dp=1ex,left]{title in head/foot}%
    \usebeamerfont{title in head/foot} a.tikhonova@g.nsu.ru
  \end{beamercolorbox}%
  \begin{beamercolorbox}[wd=.5\paperwidth,ht=2.25ex,dp=1ex,right]{date in head/foot}%
    \usebeamerfont{date in head/foot}\insertshortdate{}\hspace*{2em}
    \insertframenumber{} / \inserttotalframenumber\hspace*{2ex} 
  \end{beamercolorbox}}%
  \vskip0pt%
}
\makeatother

\begin{document}

\begin{frame}
\titlepage
\end{frame}

\begin{frame}
\frametitle{Cпектральные данные атомных систем}
    Использование спектральных данных:
    \begin{itemize}
        \item исследования астрофизики
        \item исследование продуктов сгорания
        \item в области управляемого термоядерного синтеза
    \end{itemize}
    Способы представления:
    \begin{itemize}
        \item табличный
        \item графический (схема энергетических уровней и переходов)
    \end{itemize}
\end{frame}

\begin{frame}
\frametitle{Эволюция графического представления спектральных данных}
    \begin{itemize}
        \item Спектрограммы
        \item Графическое изображение стационарных состояний по Н. Бору\\\begin{center} {$\Downarrow$} \end{center}
        \item Схема термов и переходов по Д. С. Рождественскому\\\begin{center} {$\Downarrow$} \end{center}
        \item Диаграммы Гротриана, В. Гротриан (1924г.)\\\begin{center} {$\Downarrow$} \end{center}
        \item Поиск новых графических представлений (исследование сложных спектров редкоземельных и трансурановых элементов)
        \item ИС по атомной спектроскопии, позволяющие на основе БД строить графические представления атомных систем (ASD NIST и ИС "ЭСА"(ИАиЭ))
    \end{itemize}
\end{frame}

\begin{frame}
\frametitle{Цели и задачи}
    Цель: разработка новых способов графического представления электронной структуры атомных систем.
     \\~\\
    Задачи:
    \begin{itemize}
        \item Анализ существующих графических представлений и выявление их недостатков
        \item Разработка оригинального графического инстумента для анализа атомных спектров
        \item Добавения возможности взаимодействия пользователя с диаграммой
        \item Анализ особенностей представления спектральных данных на новой диаграмме
    \end{itemize}
\end{frame}

\begin{frame}
\frametitle{Недостатки классических графических представлений атомных спектров}
    Спектрограммы: представлены только переходы без энергетических уровней
     \\~\\
    Диаграммы Гротриана: потеря читаемости при больших Z – большое число уровней и пересечений спектральных линий
\end{frame}

\begin{frame}
\frametitle{Квантограмма - оригинальная диаграмма спектральных данных}
    Основная идея: переходы отображаются точками на плоскости, координаты которых - значения энергии верхнего и нижнего уровней перехода.
    \begin{itemize}
        \item Автоматическая генерация квантограмм на основе БД для всех элементов периодической таблицы и их ионов
        \item Отображение мультиплетности, интенсивности и длины волны перехода
    \end{itemize}
\end{frame}

\begin{frame}
\frametitle{Реализация}
    \begin{itemize}
        \item Интерактивное web-приложение (Rich Internet Application)
        \item Трехуровневая клиент-серверная архитектура (MS SQL + Apache и php скрипты + WEB браузер)
        \item Библиотеки: Chart.js, JQuery
        \item Web-интерфейс:JavaScript, HTML, CSS, SVG
    \end{itemize}
    Вид квантограммы:
    \begin{itemize}
        \item 2D-диаграмма
        \item Переход - точка на графике, отображаемая символом
        \item Символ перехода соответствует мультиплетности, 
       \\~цвет - длине волны
    \end{itemize}
\end{frame}

\begin{frame}
\frametitle{Интерактивность квантограмм}
    \begin{itemize}
        \item Масштабирование
        \item Изменение единиц измерения энергии
        \item Режим отображения интенсивности перехода прозрачностью
        \item Всплывающие подсказки при наведении на переход 
        \item Изменение режима отображения переходов (изменение осей)
        \item Отображение всех компонент мультиплета в отдельном окне квантограммы при выборе одной из компонент
        \item Режим раскрашивания переходов по принадлежности к мультиплетам
        \item Фильтрация переходов по длине волны
        \item Отображение ширины конфигурации?????
    \end{itemize}
\end{frame}

\begin{frame}
\frametitle{Диаграмма уровней}
    /Не знаю где именно про нее расказывать. Есть идея рассказать о ней после того, как расскажу что я добавила отображение ширины конфигураций на диаграмму, а т.к. они пересекались, то пришла идея создать новую диаграмму только по уровням./
\end{frame}

\begin{frame}
\frametitle{Результаты }
    /Про особенности отображения спектральных данных на квантограмме/
\end{frame}

\begin{frame}
\frametitle{Итоги }
  
\end{frame}

\end{document}
