\documentclass{beamer}
\usepackage[utf8]{inputenc}
\usepackage[english,russian]{babel}
\usetheme{Madrid}
\usepackage{graphicx}
\graphicspath{{imgs/}}

\title{Разработка когнитивных способов представления спектральных данных атомных систем на основе интерактивной компьютерной графики}
\author{Тихонова А. А. \\
        Научный руководитель: Казаков В. В.}
\institute{Новосибирский Государственный Университет}

\makeatletter
\setbeamertemplate{footline}
{
  \leavevmode%
  \hbox{%
  %\begin{beamercolorbox}[wd=.333333\paperwidth,ht=2.25ex,dp=1ex,center]{author in head/foot}%
  %  \usebeamerfont{author in head/foot}Каня Кирилл, НГУ
  %\end{beamercolorbox}%
  \begin{beamercolorbox}[wd=.5\paperwidth,ht=2.25ex,dp=1ex,left]{title in head/foot}%
    \usebeamerfont{title in head/foot} a.tikhonova@g.nsu.ru
  \end{beamercolorbox}%
  \begin{beamercolorbox}[wd=.5\paperwidth,ht=2.25ex,dp=1ex,right]{date in head/foot}%
    \usebeamerfont{date in head/foot}\insertshortdate{}\hspace*{2em}
    \insertframenumber{} / \inserttotalframenumber\hspace*{2ex} 
  \end{beamercolorbox}}%
  \vskip0pt%
}
\makeatother

\begin{document}

\begin{frame}
\titlepage
\end{frame}

\begin{frame}
\frametitle{Cпектральные данные атомных систем}
    Использование спектральных данных:
    \begin{itemize}
        \item исследования астрофизики
        \item исследование продуктов сгорания
        \item в области управляемого термоядерного синтеза
    \end{itemize}
    Способы представления:
    \begin{itemize}
        \item табличный
        \item графический (схема энергетических уровней и переходов)
    \end{itemize}
\end{frame}

\begin{frame}
\frametitle{Эволюция графического представления спектральных данных}
    \begin{itemize}
        \item Спектрограммы
        \item Графическое изображение стационарных состояний по Н. Бору\\\begin{center} {$\Downarrow$} \end{center}
        \item Схема термов и переходов по Д. С. Рождественскому\\\begin{center} {$\Downarrow$} \end{center}
        \item Диаграммы Гротриана, В. Гротриан (1924г.)\\\begin{center} {$\Downarrow$} \end{center}
        \item Поиск новых графических представлений (исследование сложных спектров редкоземельных и трансурановых элементов)
        \item ИС по атомной спектроскопии, позволяющие на основе БД строить графические представления атомных систем (ASD NIST и ИС "ЭСА"(ИАиЭ))
    \end{itemize}
\end{frame}

\begin{frame}
\frametitle{Цели и задачи}
    Цель: разработка новых способов графического представления электронной структуры атомных систем.
     \\~\\
    Задачи:
    \begin{itemize}
        \item Анализ существующих графических представлений и выявление их недостатков
        \item Разработка оригинального графического инстумента для анализа атомных спектров
        \item Добавения возможности взаимодействия пользователя с диаграммой
        \item Анализ особенностей представления спектральных данных на новой диаграмме
    \end{itemize}
\end{frame}

\begin{frame}
\frametitle{Недостатки классических графических представлений атомных спектров}
    Спектрограммы: представлены только переходы без энергетических уровней
     \\~\\
    Диаграммы Гротриана: потеря читаемости при больших Z – большое число уровней и пересечений спектральных линий
\end{frame}

\begin{frame}
\frametitle{Квантограмма - оригинальная диаграмма спектральных данных}
    /основные идеи построения квантограммы/
\end{frame}

\begin{frame}
\frametitle{Реализация}
    /архитектура(клиентская и серверная часть), библиотеки/
\end{frame}

\begin{frame}
\frametitle{Интерактивность квантограмм}
    /Здесь об инструметах работы с квантограммой/
\end{frame}

\begin{frame}
\frametitle{Диаграмма уровней}
    /Не знаю где именно про нее расказывать. Есть идея рассказать о ней после того, как расскажу что я добавила отображение ширины конфигураций на диаграмму, а т.к. они пересекались, то пришла идея создать новую диаграмму только по уровням./
\end{frame}

\begin{frame}
\frametitle{Результаты }
    /Про особенности отображения спектральных данных на квантограмме/
\end{frame}

\begin{frame}
\frametitle{Итоги }
  
\end{frame}

\end{document}
