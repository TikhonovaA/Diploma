\documentclass[a4paper, 12pt]{article}
\input{preamble.tex}
\let\stdsection\section
\renewcommand\section{\newpage\stdsection}
\newcommand{\anonsection}[1]{\section*{#1}\addcontentsline{toc}{section}{#1}}
\usepackage{listings}
\begin{document}
\input{title.tex}
\newpage

\tableofcontents
\newpage

\section*{Введение}
\addcontentsline{toc}{section}{Введение}
\subsection*{Актуальность}
    
    В 1859 г. Г. Кирхгофом и Р. Бунзеном были открыты оптические спектры атомов. С этого момента в науке появился спектральный анализ - физический метод дистанционного определения химического состава вещества.\par
    Сегодня со спектральными данными атомных систем работает широкий круг специалистов, занятых анализом составов и состояний самых разнообразных физических систем. Спектральные данные могут быть зафиксированы в табличной или графической форме. Справочники и базы данных содержат информацию о сотнях нейтральных атомов и ионов и тысячах возможных энергетических уровней и переходов между ними в основном в табличном виде. Но анализ таких обширных данных доступен только опытным специалистам. Поэтому для общей ориентировки в электронном строении атомной системы используются графические представления спектральных данных. Несмотря на многолетний путь развития, нельзя сказать, что существует канонический вид графического представления данных об уровнях и переходах в многоэлектронных атомах. В настоящее время широко используются такие графические представления, как спектрограммы и диаграммы Гротриана, имеющие различные модификации.\par
    \iffalse
        В экологии атомная спектроскопия используется для мониторинга состояния окружающей среды посредством анализа загрязняющих соединений. В геологии исследуется химический состав руд и минералов для контроля качества в процессе обогащения, а также изучается метеоритный материал для определения состава космических объектов. В металлургии атомная спектроскопия также является мощным инструментом при сортировке и анализе металлического лома, ведении плавки и получении новых материалов. Также спектральные данные атомных систем необходимы при исследованиях в области астрофизики, управляемого термоядерного синтеза и других отраслях науки и техники.
        \par
        Спектральные данные могут быть зафиксированы в табличной или графической форме. Справочники и базы данных содержат информацию о переходах и энергетических уровнях атомов и ионов в основном в табличном виде. Но анализ таких обширных данных доступен только опытным специалистам. Поэтому для общей ориентировки в электронном строении атомной системы используются графические представления спектральных данных. Качественное представление таких данных обеспечивает более эффективные механизмы исследования спектров, дает информацию об электронном строении атома в наглядной форме, удобной для обработки и размышления, а также нередко позволяет увидеть зависимости и закономерности, скрытые в численных значениях спектральных параметров.  \fi\par
    
    Построение диаграмм вручную является довольно трудоемким процессом. С развитием цифровых технологий появились информационные системы по атомной спектроскопии, с возможностью автоматического построения диаграмм атомных систем на основе спектральных баз данных. Среди таких систем выделяется информационная система «Электронная структура атомов». Она отличается наиболее развитыми средствами графического представления атомных спектров и пользовательским интерфейсом. Данная работа является частью дальнейшего развития этой системы. Реализованные в информационных системах диаграммы претерпели лишь малые изменения. Однако, возможности современного веба позволяют экспериментировать с формами представления спектральных данных и интерактивными средствами взаимодействия с пользователем.
    
        \iffalseДанная работа посвящена разработке нового способа графического отображения структуры электронных переходов в атоме, основанного на компьютерных способах обработки и представления информации. В данной работе описываются особенности представления спектральных данных на полученной диаграмме. Также здесь приведен обзор различных способов графического представления спектральных данных атомных систем и информационных систем в области атомной спектроскопии.\fi
    
    \subsection*{Цели и задачи исследования}
    Используя методы когнитивной графики можно получить оригинальное и более полное графическое представление спектральной информации об уровнях и переходах в атоме удобным для восприятия человеком способом. Такое представление обеспечивает более эффективные механизмы исследования спектров за счет наглядного проявления некоторых закономерностей и физических свойств отображаемых данных. Целью работы является разработка новых когнитивных способов графического представления электронной структуры атомных систем на основе интерактивной компьютерной графики.\par 
    Для достижения цели были поставлены следующие задачи:
    \begin{itemize}
        \item Анализ существующих графических представлений спектральных данных атомных систем
        \item Проектирование оригинального графического инструмента для анализа атомных спектров
        \item Разработка оригинального графического инструмента для анализа атомных спектров
        \item Тестирование и эксплуатация.
    \end{itemize}
    
    \subsection*{Научная новизна и практическая ценность}
    
    \subsection*{Научная апробация}
    Выступление на МНСК - публикация: Тихонова А.А. Разработка техники квантограмм как оригинальных диаграмм электронной
структуры атомных систем и создание программы их генерации // Информационные
технологии : Материалы 57-й Междунар. науч. студ. конф. 14–19 апреля 2019 г. / Новосиб. гос. ун-т. — Новосибирск : ИПЦ НГУ, 2019, с. 92
Выступление DICR-2019. Публикации: Казаков В.В., Казаков В.Г., Мешков О.И., Тихонова А.А. СРЕДСТВА ВИЗУАЛИЗАЦИИ И АНАЛИЗА СПЕКТРОВ ИНФОРМАЦИОННОЙ СИСТЕМЫ "ЭЛЕКТРОННАЯ СТРУКТУРА АТОМОВ": Труды XVII Международной конференци DICR-2019, Новосибирск, 3-6 декабря 2019 г: ИВТ СО РАН, 2019
Kazakov, V. V., Kazakov, V. G., Meshkov, O. I., Tikhonova, A. A. (2019). Visualization and analysis tools of the spectra of the Information System «Electronic Structure of Atoms». CEUR Workshop Proceedings, 2569, 48-54.
    

\section{Графические представления спектральных данных атомных систем}
 Ввиду сложности атомных систем, неотъемлемой частью их анализа является качественное представление спектральных данных. Графические представления спектральных данных обеспечивают более эффективные механизмы исследования спектров, дают информацию об электронном строении атома в наглядной форме, удобной для обработки и размышления, а также нередко позволяют увидеть зависимости и закономерности, скрытые в численных значениях спектральных параметров.
    \subsection{Эволюция графического представления спектральных данных}
    (здесь будет немного теории про дискретность уровней, термы и квнтовые числа, стационарные состояния и т.д.)
    Ввиду необычности, непривычности квантовых законов, дискретного характера величин эволюция графической версии состояний атомов оказалась непростой.\par
    (добавить спектрограммы)
    В 1923 г. Н. Бор предложил графическое изображение стационарных состояний атомной системы (рис. 1) [Бор Н. Три статьи о спектрах и строении атомов (М. - Пг.: Госиздат, 1923]. Здесь стационарные состояния обозначены точками, а переходы между ними, возникающие при обычных условиях возбуждения, показаны стрелками. Для классификации состояний атома используются только два квантовых числа n и k - главное и орбитальное квантовые числа. Расстояние от точки до вертикальной линии aa пропорционально энергии данного состояния.\par
    Примерно в это же время (1921 г.) Д. С. Рождественский в [Рождественский Д.С. Значение спектральных серий.] представил схему спектральных линий для атома Na (рис. 2). На схеме все известные термы по горизонтали распределены в столбцы с одинаковым орбитальным квантовым числом, а по вертикали в соответствии с увеличением главного квантового числа без соблюдения масштаба по шкале энергий. Переходы между состояниями, которым здесь уделено особое внимание, указаны стрелками.\par
    В 1924 г. В. Гротриан представил диаграмму, в которой для классификации состояний использовалось третье квантовое число j - внутреннее, описывающее прецессию вращения плоскости орбиты. Переход возможен при изменении j на 0 или $\pm$1. На диаграмме Гротриана (рис. 3) уровни энергии, между которыми происходят радиационные переходы, распределены по рядам, называемым энергетическими лестницами. Каждому ряду соответствует свое азимутальное число k. Энергетическая ось направлена вверх и расстояние между ступенями лестницы убывает по мере возрастания энергии (увеличение главного квантового числа n). Ступени мультиплетных спектров распадаются на несколько близких ступеней. В книге [Grotrian W Graphische Darstellung der Spektren von Atomen und Ionen mit ein, zwei und drei Valenzelektronen (Berlin: J. Springer, 1928)] В. Гротриан дополнил ранее представленную диаграмму линиями, обозначающими оптически разрешенные переходы. Более интенсивные переходы отображены более толстыми линиями и на всех линиях указаны значения длины волны в ${\buildrel _{\circ} \over {\mathrm{A}}}$. Такие диаграммы оказались наиболее наглядными и удобными для представления на одной странице, что сделало их стандартом для графического представления спектров атомных систем.\par
    Дальнейшее развитие спектроскопии спровоцировало разработку диаграмм Гротриана для всех атомов элементов периодической системы и их ионов. Возникали сложности при отображении спектров атомов с большим значением заряда ядра Z - необходимо было указывать все электронные конфигурации и состояния атомного остатка. Для этого требовалось проводить отбор наиболее важных переходов, поэтому диаграммы Гротриана не могли содержать полную информацию об атомных спектрах. Это спровоцировало поиск новых решений и использование модифицированных диаграмм Гротриана. Выбор используемых диаграмм определялся практическими задачами, такими как изучение редкоземельных и трансурановых элементов, исследования физики лазеров, астрофизики и вакуумного ультрафиолета.
    
    \subsection{Информационные системы в области атомной спектроскопии}
    Печатные издания в области атомной спектроскопии не всегда содержат актуальные данные и информация в них всегда отобрана и упорядочена определенным способом. Поэтому они могут не соответствовать требованиям конкретного специалиста. В настоящее время эффективную работу с данными атомных спектров обеспечивают информационные системы (ИС) в области спектроскопии, построенные на основе базы данных. Их преимущество состоит в возможности пополнения баз данных актуальными данными последних исследований и удобстве представления данных конкретному пользователю за счет наличия механизмов выборки и сортировки спектральных данных. Графические представления в ИС более удобны для пользователей, так как могут генерироваться автоматически по базе данных, сопровождаясь инструментами настройки отображения, фильтрации и выбора данных. Также за счет возможности изменения масштаба электронных диаграмм можно получить более детальную информацию о спектре. Наиболее популярными являются информационные ресурсы, опубликованные в Интернет, так как предоставляют пользователям легкий доступ к самым актуальным данным.\par
    В настоящее время в мире существуют около 20 ИС по спектральным данным атомных систем. Большинство из них предоставляют файлы с данными о всех переходах и уровнях атома без возможности сортировки, фильтрации и поиска нужной информации по базе данных. Далее кратко описаны наиболее популярные базы данных по спектрам атомных систем, которые имеют собственный веб-интерфейс и инструменты для визуализации спектральных данных по БД.\par
    NIST Atomic Spectra Database (NIST ASD) - ИС по атомной спектроскопии, поддерживается группой атомной спектроскопии подразделения атомной физики физической лаборатории NIST (Национальный институт стандартов и технологии, США) [Kramida, A., Ralchenko, Yu., Reader, J. and NIST ASD Team (2019). NIST Atomic Spectra Database (version 5.7.1). Available: https://physics.nist.gov/asd [Sun May 17 2020]. National Institute of Standards and Technology, Gaithersburg, MD.]. В БД системы хранятся экспериментальные и теоретические данные о 111 тыс. уровней энергии 56 элементов и 271 тыс. радиационных переходов 98 элементов. Данная система предоставляет информацию о спектрах атомных систем в табличной и графической форме с возможностью отбора и сортировки уровней и переходов.  Спектрограммы и диаграммы Гротриана NIST ASD строятся для большинства атомных систем. Диаграммы NIST ASD отображаются в виде динамического HTML-документа без первоначального отбора спектральных линий (рис. 4а), а спектрограммы в виде черно-белого изображения в форматах png, pdf, eps (рис. 4б). Для больших значений заряда ядра на диаграммах отображаются уровни без переходов, а для элементов с атомным номером более 92 диаграммы не строятся.\par
    Информационная система «Электронная структура атомов» (ИС ЭСА) - ИС по спектральным данным атомов и ионов, разрабатывается совместно специалистами НГУ и ИАиЭ СО РАН с 2003 г. [ ]. База данных ЭСА хранит информацию о 109 тыс. уровней 115 атомов и ионов и 196 тыс. переходов 102 атомов и ионов. ИС ЭСА предоставляет спектральную информацию в табличной и графической форме в виде динамических HTML-документов. Особенностью системы является развитый пользовательский интерфейс с набором инструментов для поиска и фильтрации данных и интерактивной работы с графическими представлениями, а также наличие уникальных способов визуализации. Основными графическими инструментами, представленными в ИС ЭСА, являются интерактивные цветные спектрограммы с визуальным отображением длин волн и интенсивностей спектральных линий и диаграммы Гротриана, использующие алгоритм автоматического отбора отображаемых переходов (рис. 5) [Казаков В. Г., Казаков В. В., Жакупов М. Б., Яценко А. С. Задача автоматического построения диаграмм атомных спектров и опыт ее решения в ИС ЭСА // Вестник НГУ. Серия: Информ. технологии. 2010. Т. 8, № 3. С. 66‒78.]. В системе имеется инструмент сравнительного анализа экспериментально полученного спектра атомной системы с эталонным, не имеющий аналогов в других Интернет-ресурсах в области атомной спектроскопии. Также в  ИС ЭСА развиваются когнитивные диаграммы атомных спектров, например, реализована круговая спектральная диаграмма [Научная и когнитивная графика в информационных системах по атомной спектроскопии. В.В. Казаков, В.Г. Казаков, О.И. Мешков, К.Б. Жумадилов]. Данная работа является частью дальнейшего развития этой системы.
    
    \subsection{Анализ существующих графических представлений}
    В настоящее время для анализа спектров атомных систем широко используются спектрограммы и диаграммы Гротриана.\par
    Диаграмма Гротриана, размещенная на одной странице, дает информацию об электронном строении атома в наглядной форме. Уровни и переходы отображаются отрезками и линиями на координатной плоскости. При попытке размещения всех известных уровней атомной системы и переходов между ними на диаграмме она становится практически нечитаемой из-за сплошной сетки пересекающихся линий. Эта проблема особенно видна на диаграммах для атомов с большим значением заряда ядра Z - чем больше значение Z, тем больше возможных энергетических состояний и переходов. При отрисовке диаграмм вручную или с применением специальных алгоритмов для построения электронных диаграмм производится отбор наиболее значимых уровней и переходов. Это число в среднем ограничено 20-25 линиями [ ]. Также при обычном масштабе в десятки тысяч см$^{-1}$ на диаграмме сложно разместить мультиплетные уровни соблюдая масштаб, так как при расщеплении энергии уровней отличаются всего на сотни или на десятки см$^{-1}$. (сложны для понимания неподготовленному пользователю)\par
    Спектрограммы содержат информацию о переходах, которые отображаются вертикальными линиями, расположенными по оси X в соответсвии со значением длины волны (ссылка на рисунок). Недостатком является то, что переходы представлены без информации об энергетических уровнях между которыми он происходит. Также, в случае если спектрограмма не имеет высокого разрешения, линии тонкого и сверхтонкого расщепления сливаются в одну. То же происходит при размещении большого количества спектральных линий. На спектрограмме сложно увидеть разделение спектра атома на серии и мультиплеты.
\section{Архитектура и проектирование}
    \subsection{Проектирование оригинальной диаграммы атомных спектров}
    Проанализировав существующие графические представления спектральных данных, была сформулирована основная идея построения новой диаграммы: переходы отображаются точками на плоскости, координатами которых являются значения энергии верхнего и нижнего уровней перехода.
    \parК диаграмме были предъявлены следующие требования, обеспечивающие ее когнитивность:
    \begin{itemize}
        \item диаграммы должны быть интегрированы в существующую информационную систему ЭСА;
        \item диаграммы должны автоматически строиться для всех атомов и ионов по базе данных ИС ЭСА, для которых в БД существуют данные о переходах;
        \item на диаграмме должна отображаться информация о мультиплетности, длине волны, интенсивности перехода и уровнях между которыми он происходит;
        \item диаграммы должны быть интеракивными и сопровождаться инструментами настройки отображения контента, фильтрации переходов и инструментами масштабирования.
    \end{itemize}\par
    Для более эффективного восприятия спектральной информации необходимо визуализировать различные физичекие характеристики переходов таким образом, чтобы зрительные образы не накладывались друг на друга и не загромождали диаграмму. Далее перечислены решения, которые будут использоваться для отображения параметров переходов:
    \begin{itemize}
        \item каждый переход отображается на диаграмме символом;
        \item каждому значению мультиплетности перехода сопоставлена форма символа;
        \item цвет символа указывает на значение длины волны перехода;
        \item интенсивность перехода отображается прозрачностью символа.
    \end{itemize}\par
    \subsection{Архитектура приложения}
    Разрабатываемое приложение должно интегрироваться и функционировать с реализованной частью информационной системы ЭСА. Данная ИС основана на базе данных спектров атомов и ионов и имеет традиционную для таких систем трехуровневую клиент-серверную архитектуру (рис. ). Система состоит из трех блоков: клиент, сервер приложений (промежуточный уровень) и сервер баз данных (серверный уровень).\par
    Клиентский уровень представлен web-браузером на компьютере конечного пользователя. Основное предназначение программного обеспечения (ПО) клиентского уровня - отображение web-страниц по запросу пользователя. ПО отображает HTML-документы, отформатированные с помощью стилей CSS и выполняет клиентские скрипты.\par
    В роли сервера баз данных выступает система управления базами данных (СУБД), реализованная в ИС ЭСА СУБД MS SQL Server. ПО сервера баз данных обеспечивает хранение и предоставление численных значений спектральных данных атомных систем.\par
    Промежуточный уровень состоит из web-сервера Apache и сервера приложений. Программное обеспечение, выполняющее функции web-сервера принимает запросы и выдает ответы клиенту по протоколу HTTP. Сервер приложений представляет собой комплект PHP-скриптов. Его основная функция состоит в генерации динамических web-страниц. Также сервер приложений является связующим звеном между клиентским ПО и СУБД. На сервере формируются SQL-запросы к СУБД и происходит обработка полученных данных для передачи клиенту.\par
    Для обеспечения интерактивности диаграмм было решено реализовать веб-приложение по принципу "толстого" клиента (рис. ). Серверная часть этого приложения выполняется на сервере и отвечает за формирование начальной веб-страницы и обработку запросов на получение новых данных из БД, необходимых для построения диаграмм на клиентском ПО. Дальнейшая работа по обработке данных, построению диаграмм и обеспечению их интерактивности ложится на клиентскую часть.\par
    Данное решение имеет ряд преимуществ перед традиционными web-приложениями:
    \begin{itemize}
        \item снижение нагрузки на сервер за счет выполнения вычислений на клиентском ПО, при запросе данных сервер отправляет только новые данные из БД, а обновляет и формирует HTML-код страницы браузер клиентского компьютера;
        \item удобная работа с приложением в случае низкой скорости интернет-соединения, так как на клиент передается минимальный поток данных, необходимый для начала работы приложения;
        \item широкая функциональность приложения и более гибкий интерфейс (интерактивность).
    \end{itemize}
    Однако, производительность такого приложения напрямую зависит от производительности клиентского ПО.\par
    ИС ЭСА обладает классической для подобных баз данных моделью данных атомных спектров (рис. ). Спектральные данные атомных систем представлены в БД ИС ЭСА следующими структурами:
    \begin{itemize}
        \item таблица элементов периодической системы (periodic table);
        \item таблица атомов и ионов (atom);
        \item таблица энергетических уровней (level);
        \item таблица переходов (transition).
    \end{itemize}\par
    Приложение по построению диаграмм, имеющее такую архитетуру, может быть интегрировано в любую подобную ИС по атомной спектроскопии с минимальными изменениями кода. Необходимо изменить код серверной части под модель другой ИС, а остальной код останется без изменений.
    \subsection{Средства разработки веб-приложения}
    Для совместимости с реализованной частью ИС в качестве языка для написания серверной части приложения был выбран язык PHP. Скрипты, написанные на PHP, широко применяются для разработки веб-приложений, поскольку имеют широкий выбор встроенных инструментов разработки.\par
    Для разработки веб-интерфейса, т.е. клиентской части приложения, были выбраны стандартные для этой задачи языки, которые поддерживают все современные браузеры: язык разметки HTML, язык таблиц стилей CSS для оформления внешнего вида страницы, язык сценариев JavaScript для создания интерактивного пользовательского веб-интерфейса.\par
    Для отрисовки диаграммы была выбрана популярная JavaScript библиотека визуализации данных Chart.js, предназначенная для создания графиков и диаграмм. Для отрисовки графиков она использует Canvas элемент HTML5 и имеет отличную производительность рендеринга во всех современных браузерах. Библиотека обладает простым API и позволяет строить 8 типов графиков и совмещать разные типы. Графики адаптивны, то есть перерисовываются при изменении размера окна.Также в библиотеке встроены средства работы с анимацией, что позволяет эффектно видоизменять графики.\par
    В реализации пользовательского интерфейса была также задействована технология векторной графики SVG. Использование данной технологии удобно тем, что SVG элементы являются элементами DOM (объектная модель документа). 	Вследствие чего их отображение можно изменять, применяя свойства CSS, а поведением управлять с помощью скриптов JavaScript.
\section{Реализация}
  \subsection{Реализация приложения по построению диграмм атомных спектров}
  Для построения диаграмм необходимо получить спектральные данные выбранной атомной системы. Эти данные хранятся в базе данных ИС ЭСА. Программу можно разделить на 3 части, взаимодействующие между собой: сервер ИС ЭСА, сервер приложения построения диаграмм и клиентская часть приложения. Для передачи данных о спектрах из ИС ЭСА на клиент был написан серверный PHP-скрипт.\par
  Для получения данных для построения диаграммы клиентский скрипт осуществляет отправку AJAX-запроса на сервер HTTP-методом GET в обертке JavaScript библиотеки jQuery. Фрагмент кода AJAX-запроса с указанием аббривеатуры и ионизации запрашиваемого атома в качестве параметров запроса:
    \begin{lstlisting}
    $.ajax({
            url:"getdata.php",
            type: "get",
            dataType: "json",
            data:{
                "ABBR": abbr,
                "IONIZATION": ion,
            },
            success:function(data) {} 
    });
    \end{lstlisting}
    При использовании такого решения не происходит полная перезагрузка страницы. Использование AJAX позволяет асинхронно подгружать только новые данные с сервера, необходимые для построения диаграммы, а не загружать страницу полностью. Таким образом, сервер формирует код начальной веб-страницы, а дальнейшим обновлением и формированием HTML-кода страницы занимается веб-браузер клиента. Это позволяет значительно сократить время загрузки страницы и снижает нагрузку на сервер.
  \par
  Взаимодействие между скриптом и сервисом ЭСА происходит по принципам REST. ИС ЭСА предоставляет API для получения данных об атомах, переходах и уровнях атомных систем. Сервер приложения отправляет GET-запросы с помощью функций встроенной в PHP библиотеки libcurl. При получении GET-запроса сервер ИС запрашивает нужные данные из БД и отправляет ответ в фрмате JSON на сервер разрабатываемого приложения. Обмен данными происходит по протоколу HTTP. После получения необходимых данных об атоме, его уровнях и переходах сервер приложения группирует их нужным образом и отправляет ответ клиенту в формате JSON. Данные группируются следующим образом:
  \begin{lstlisting}
    {atom: {atom info}, 
    transitions: [{transition info, 
                lower level info, 
                upper level info}, ...],
    levels: [{level info}, ...]
    }
    \end{lstlisting}
  \par
  После получения ответа клиентский скрипт обрабатывает данные и формирует массивы данных, необходимые для построения диаграмм.
  (chart.js)
  \subsection{Элементы веб-интерфейса}
  как происходит взаимодействие с пользователем
\section{Тестирование и эксплуатация}


\section*{Заключение}
\addcontentsline{toc}{section}{Заключение}
    \parВ рамках данной работы был проведен анализ существующих графических представлений спектральных данных атомных систем, таких как спектрограммы и диаграммы Гротриана, и выявлены их недостатки. Разработана оригинальная диаграмма электронных переходов в атоме - квантограмма.  Данные диаграммы были реализованы в виде web-приложения, реализующего на клиентской стороне интерактивную работу с данными. Реализован ряд опций отображения и фильтрации спектральных данных. Также в качестве дополнительного функционала была разработана диаграмма уровней. Проведен анализ представления спектральных данных на квантограммах и перечислены особенности их представления. (размышления на тему применимости квантограмм в обучениии и т.д. и т.п.)

\section*{Список литературы}
\addcontentsline{toc}{section}{Список литературы}

\end{document}
